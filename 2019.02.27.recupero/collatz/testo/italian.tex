\makeatletter
\renewcommand{\this@inputfilename}{\texttt{stdin}}
\renewcommand{\this@outputfilename}{\texttt{stdout}}
\makeatother

%\introduzione{Olimpiadi Italiane di Informatica - Selezioni Territoriali 2014}

\medskip

Consideriamo il seguente procedimento, che prende in ingresso un intero positivo $N$:
\begin{enumerate}
\item Se $N$ vale $1$, termina.
\item Se $N$ è pari, passa ad $N := N/2$ e ripeti il procedimento.
\item Se $N$ è dispari, passa ad $N := 3N +1$ e ripeti il procedimento.
\end{enumerate}

Ecco la traccia dei numeri naturali visitati se inizialmente partiamo dal valore $N=6$:
$$6, 3, 10, 5, 16, 8, 4, 2, 1.$$

La congettura di Collatz, chiamata anche congettura $3N+1$, afferma che il procedimento 
descritto qui sopra termina sempre, qualsiasi sia il naturale $N$ di partenza.

È riferendosi a questa famosa congettura che il celebre matematico Erdős commentò 
sul come questioni semplici ma elusive mettono in evidenza quanto poco noi si possa 
accedere ai misteri del ``Grande Libro''.

Ti chiediamo di scrivere una funzione che riceve in input un numero naturale $N$ 
e ritorna la lista dei numeri naturali visitati applicando il procedimento in questione partendo da $N$.

Trovi un template della soluzione nel file \textbf{collatz\_template\_sol.py} tra 
gli allegati, dovrai solo curare l'implementazione della funzione richiesta:

\begin{verbatim}
def simulate(n):
    confs = []
    confs.append(5)
    confs.append(16)
    confs.append(8)
    confs.append(4)
    confs.append(2)
    confs.append(1)
    return confs
\end{verbatim}

Attualmente {\tt simulate} restituisce la lista corretta solo quando le viene passato 
il numero naturale $N=5$. Il tuo compito è correggere/completare l'implementazione 
attuale della funzione {\tt simulate}. 

\textbf{NOTA}: per aggiungere un elemento in coda ad una lista va utilizzato 
il metodo \texttt{append()}, come nell'esempio sopra.

\InputFile
Il programma riceve come suo input un numero naturale $N$.

\OutputFile
Il programma restituisce in output la sequenza dei numeri naturali visitati applicando 
il procedimento in questione partendo da $N$, stampandoli uno per riga. 

\textbf{NOTA}: viene fornita una descrizione del formato di input/output soltanto 
per facilitarvi il test sul vostro computer. Per sottomettere il problema è obbligatorio 
utilizzare il template che potete scaricare fra gli allegati del problema, avendo 
cura di modificare solamente l'implementazione delle funzioni richieste. Questo 
è necessario per garantire la compatibilità del vostro programma con il sistema 
di valutazione, che potrebbe utilizzare una versione di python diversa (quale python2).

\Examples
\begin{example}
\exmpfile{collatz.input0.txt}{collatz.output0.txt}%
\exmpfile{collatz.input1.txt}{collatz.output1.txt}%
\end{example}

\Constraints
\begin{itemize}[nolistsep, noitemsep]
\item $2 \le N \le 1\,000 $.
\item E' noto che, per qualsiasi $N$ minore di $1000$, il numero di naturali visitati è minore di $200$.
\end{itemize}

\Scoring
\begin{itemize}
  \item \textbf{Subtask 1 [10 punti]:} gli esempi del testo.
  \item \textbf{Subtask 2 [30 punti]:} $n \leq 10$.
  \item \textbf{Subtask 3 [30 punti]:} $n \leq 100$.
  \item \textbf{Subtask 4 [30 punti]:} $n \leq 1000$.
\end{itemize}
  
