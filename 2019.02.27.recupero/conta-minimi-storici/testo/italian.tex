\renewcommand{\inputfile}{\texttt{stdin}}
\renewcommand{\outputfile}{\texttt{stdout}}
\makeatletter
\renewcommand{\this@inputfilename}{\texttt{stdin}}
\renewcommand{\this@outputfilename}{\texttt{stdout}}
\makeatother

Siamo stati incaricati di condurre un'analisi sugli andamenti degli indici azionari. 
Una lista di numeri naturali può essere utilizzata per codificare una sequenza di 
naturali, che nelle nostre applicazioni rappresenta una generica serie temporale. 
Si consideri ad esempio la seguente sequenza

\[
\mathbf{15}, 22, 18, \mathbf{14}, \mathbf{12}, 13, 19, 21, 25, \mathbf{7}, 7, 10, 17
\]
  
dove abbiamo riportato in neretto quei numeri che sono minimi storici della serie 
temporale rappresentata.
Diamo ora una rappresentazione più fine, 
dove distinguiamo i minimi storici in funzione della parità del loro valore come numeri naturali ($0$=pari, $1$=dispari):

\[
\underbrace{15}_{1}, 22, 18, \underbrace{14}_0, \underbrace{12}_0, 13, 19, 21, 25, \underbrace{7}_1, 7, 10, 17
\]
  
\noindent
Devi dare corretta implementazione alle seguenti due funzioni:\\

\begin{verbatim}
      conta_minimi_storici_pari(st_list)
\end{verbatim}

\noindent
e

\begin{verbatim}
      conta_minimi_storici_dispari(st_list)
\end{verbatim}

\noindent
Entrambe ricevono in input una serie temporale codificata come una lista di numeri naturali. 
Esse intendono restituire in output il numero di minimi storici di valore pari e dispari, rispettivamente.

    
Trovi un template della soluzione nel file \textbf{\file{conta-minimi-storici\_template\_sol.py}} 
tra gli allegati, dovrai solamente fornire l'implementazione delle due funzioni 
di modo che resituiscano sempre la risposta corretta.

Si noti che il primo elemento della serie semporale, ossia {\tt st\_list[0]}, 
è sempre un minimo storico, infatti un elemento è un minimo storico quando tutti 
gli elementi che lo precedono sono strettamente maggiori di esso. 
Ovviamente sarà la parità di {\tt st\_list[0]} a stabilire se esso debba essere 
conteggiato come minimo storico di valore pari oppure dispari. 

\InputFile
Il vostro programma, nell'ordine, riceve in input:
\begin{itemize}
\item un numero naturale $n$, indicante la lunghezza della sequenza considerata 
\item un numero naturale $p$, indicante la funzione da chiamare
\item una sequenza di $n$ numeri naturali, i valori di cui calcolare i minimi storici
\end{itemize}

\OutputFile
\begin{itemize}
\item se $p = 0$ il programma restituisce in output il numero di minimi storici pari,
  ossia il valore restituito dalla funzione \texttt{conta\_minimi\_storici\_pari(st\_list)}
\item se $p = 1$ il programma restituisce in output il numero di minimi storici 
  dispari, ossia il valore restituito dalla funzione \texttt{conta\_minimi\_storici\_dispari(st\_list)}
\end{itemize}

\textbf{NOTA}: viene fornita una descrizione del formato di input/output soltanto 
per facilitarvi il test sul vostro computer. Per sottomettere il problema è obbligatorio 
utilizzare il template che potete scaricare fra gli allegati del problema, avendo 
cura di modificare solamente l'implementazione delle funzioni richieste. Questo 
è necessario per garantire la compatibilità del vostro programma con il sistema 
di valutazione, che potrebbe utilizzare una versione di python diversa (quale python2).

\Examples
% ridefinisco la dimensione delle colonne di input/output
\exmpwidinf=0.575\textwidth % default 0.475
\exmpwidouf=0.375\textwidth % default 0.475
\begin{example}
\exmpfile{conta_minimi_storici.input0.txt}{conta_minimi_storici.output0.txt}%
\exmpfile{conta_minimi_storici.input1.txt}{conta_minimi_storici.output1.txt}%
\exmpfile{conta_minimi_storici.input2.txt}{conta_minimi_storici.output2.txt}%
\exmpfile{conta_minimi_storici.input3.txt}{conta_minimi_storici.output3.txt}%
\exmpfile{conta_minimi_storici.input4.txt}{conta_minimi_storici.output4.txt}%
\end{example}

\Constraints
\begin{itemize}[nolistsep, noitemsep]
  \item $1 \le n \le 100\,000$.
\end{itemize}

\Scoring
\begin{itemize}
  \item \textbf{Subtask 1 [0 punti]:} gli esempi del testo.
  \item \textbf{Subtask 2 [20 punti]:} $n = 1$.
  \item \textbf{Subtask 3 [20 punti]:} $n \leq 100$, i valori nella serie temporale sono tutti pari.
  \item \textbf{Subtask 4 [20 punti]:} $n \leq 100$, i valori nella serie temporale sono tutti dispari.
  \item \textbf{Subtask 5 [20 punti]:} $n \leq 100$, i valori nella serie possono essere sia pari che dispari.
  \item \textbf{Subtask 6 [20 punti]:} $n \leq 100\,000$.
\end{itemize}
  
