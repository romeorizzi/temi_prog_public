\documentclass[a4paper,11pt]{article}
\usepackage{nopageno} % visto che in questo caso abbiamo una pagina sola
\usepackage{lmodern}
\renewcommand*\familydefault{\sfdefault}
\usepackage{sfmath}
%\usepackage{amsmath}
\usepackage[utf8]{inputenc}
\usepackage[T1]{fontenc}
\usepackage[italian]{babel}
\usepackage{indentfirst}
\usepackage{graphicx}
\usepackage{tikz}
\usepackage{wrapfig}
\newcommand*\circled[1]{\tikz[baseline=(char.base)]{
		\node[shape=circle,draw,inner sep=2pt] (char) {#1};}}
\usepackage{enumitem}
% \usepackage[group-separator={\,}]{siunitx}
\usepackage[left=2cm, right=2cm, bottom=3cm]{geometry}
\frenchspacing

\newcommand{\num}[1]{#1}

% Macro varie...
\newcommand{\file}[1]{\texttt{#1}}
\renewcommand{\arraystretch}{1.3}
\newcommand{\esempio}[2]{
\noindent\begin{minipage}{\textwidth}
\begin{tabular}{|p{11cm}|p{5cm}|}
	\hline
        \textbf{\file{input (da stdin)}} & \textbf{\file{output (su stdout)}}\\
%	\textbf{File \file{input.txt}} & \textbf{File \file{output.txt}}\\
	\hline
	\tt \small #1 &
	\tt \small #2 \\
	\hline
\end{tabular}
\end{minipage}
}

\newcommand{\sezionetesto}[1]{
    \section*{#1}
}


%%%%% I seguenti campi verranno sovrascritti dall'\include{nomebreve} %%%%%
\newcommand{\nomebreve}{}
\newcommand{\titolo}{}

% Modificare a proprio piacimento:
\newcommand{\introduzione}{
%    \noindent{\Large \gara{}}
%    \vspace{0.5cm}
    \noindent{\Huge \textbf \titolo{}~(\texttt{\nomebreve{}})}
    \vspace{0.2cm}\\
}

\begin{document}

\renewcommand{\nomebreve}{perm\_recognition}
\renewcommand{\titolo}{Riconoscere Permutazioni}

\introduzione{}

Siamo soliti denotare con $\mathbf{N}_n := \{0,1, \ldots, n-1 \}$ l'insieme dei primi $n$ numeri naturali.
Una funzione $f:\mathbf{N}_n \mapsto \mathbf{N}$ può essere facilmente codificata tramite una lista {\tt f}, il cui $i$-esimo elemento sia appunto {\tt f[i]} per ogni $i\in \mathbf{N}_n$.

Una permutazione è una funzione \emph{biunivoca} $f:\mathbf{N}_n \mapsto \mathbf{N}_n$.
Intuitivamente, in una permutazione il dominio ed il codominio coincidono, e gli elementi vengono solo spostati di posto (o rilabellati) ma senza che nessuno vada perso.
In particolare, abbiamo l'iniettività:
\[
   f(a) = f(b) \; \Rightarrow \; a=b
\]
e la suriettività:
\[
   \forall a\in \mathbf{N}_n  \; \exists b  \; \; \mbox{tale che}  \;f(b) = a.
\]


Devi scrivere una funzione {\tt è\_perm} che riceve in input una lista $f$.
Tale lista codifica una funzione $f:\mathbf{N}_n \mapsto \mathbf{N}$.
Vogliamo che la funzione ritorni {\tt True} se $f$ è una permutazione e {\tt False} altrimenti.\\

Trovi un template della soluzione nel file \textbf{\file{perm\_recognition.py}} tra gli attachments, dovrai solo risistemare l'implementazione della funzione richiesta che attualmente presenta degli evidenti falsi positivi:

\begin{verbatim}
def è_perm(f):
    return True
\end{verbatim}


\sezionetesto{Dati di input}
Il vostro programma riceve come suo input un numero naturale $n$ seguito da una sequenza di $n$ numeri naturali. In questo modo esso codifica una generica funzione $f:\mathbf{N}_n \mapsto \mathbf{N}$ dove il parametro $n$ è variabile.

\sezionetesto{Dati di output}

Il programma ritorna in output la stringa ``False'' se il riconoscimento ha avuto esito negativo, mentre ritorna la stringa ``True'' se la funzione $f$ esaminata è effettivamente una permutazione.

% Esempi
\sezionetesto{Esempi di input/output}
\esempio{3

1 0 2}{
True
}

\esempio{3

1 3 2}{
False
}

\esempio{3

2 1 2}{
False
}



% Assunzioni
\sezionetesto{Assunzioni}
\begin{itemize}[nolistsep, noitemsep]
  \item $0 \le n \le 1\,000\,000$.
\end{itemize}

  \section*{Subtask}
  \begin{itemize}
    \item \textbf{Subtask 1 [10 punti]:} gli esempi del testo.
    \item \textbf{Subtask 2 [30 punti]:} $n \leq 10$.
    \item \textbf{Subtask 3 [30 punti]:} $n \leq 100$.
    \item \textbf{Subtask 4 [30 punti]:} $n \leq 1\,000\,000$.
  \end{itemize}
  


\end{document}
